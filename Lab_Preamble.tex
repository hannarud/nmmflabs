\documentclass[a4paper,12pt]{report}

\usepackage[russian]{babel}
\usepackage[utf8]{inputenc}
\usepackage{amssymb,amsthm,amsmath,amscd}
\usepackage{graphicx}
\usepackage{float}
\usepackage{cite}
%пакет cite служит для простой вставки ссылок на публикации в работе, если что,
%вызов этого пакета можно закомментировать с помощью символа %

%Оформление теорем, лемм и т.д.
\theoremstyle{plain} % default
\newtheorem{Theorem}{Теорема}[chapter]
\newtheorem{Lemma}{Лемма}[chapter]
\newtheorem{Proposition}{Предложение}[chapter]
\newtheorem{Corollary}{Следствие}[chapter]
\newtheorem{Statement}{Утверждение}[chapter]

\theoremstyle{definition}
\newtheorem{Definition}{Определение}[chapter]
\newtheorem{Conjecture}{Гипотеза}[chapter]
\newtheorem{Algorithm}{Алгоритм}[chapter]
\newtheorem{Property}{Свойство}[chapter]

\theoremstyle{remark}
\newtheorem{Remark}{Замечание}[chapter]
\newtheorem{Example}{Пример}[chapter]
\newtheorem{Note}{Примечание}[chapter]
\newtheorem{Case}{Случай}[chapter]

%Определение полей
\setlength{\oddsidemargin}{0cm}% 1in=2.54см
\setlength{\hoffset}{0.46cm}% 1in+\hoffset=3cm = левое поле;

\setlength{\textwidth}{17cm}% 21cm-3cm(левое поле)-1cm(правое поле)=17cm;

\setlength{\headheight}{0cm}%
\setlength{\topmargin}{0cm}%
\setlength{\headsep}{0cm}%
\setlength{\voffset}{-0.54cm}% 1in+\voffset=2cm = верхнее поле;

\setlength{\textheight}{25.7cm}% 29.7cm-2cm(верхнее поле)-2cm(нижнее поле)=25.7cm;

%Оформление глав, разделов и т.д.
\makeatletter%

%не подавлять абзацный отступ в главах
\renewcommand{\chapter}{\cleardoublepage\thispagestyle{plain}%
\global\@topnum=0 \@afterindenttrue \secdef\@chapter\@schapter}

%оформление нумерованных глав
\renewcommand{\@makechapterhead}[1]{%Начало макроопределения
\vspace*{50pt}%Пустое место вверху страницы
{\parindent=18pt \normalfont\Large\bfseries
\thechapter{} %номер главы
\normalfont\Large\bfseries #1 \par %заголовок от текста
\nopagebreak %чтоб не оторвать заголовок от текста
\vspace{40 pt} %между заголовком и текстом
}%конец группы
}%коней макроопределения

%оформление ненумерованных глав
\renewcommand{\@makeschapterhead}[1]{%Начало макроопределения
\vspace*{50pt}%Пустое место вверху страницы
{\parindent=18pt \normalfont\Large\bfseries #1 \par %заголовок от текста
\nopagebreak %чтоб не оторвать заголовок от текста
\vspace{40pt} %между заголовком и текстом
}%конец группы
}%коней макроопределения

%оформление разделов
\renewcommand{\section}{\@startsection{section}{1}{18pt}%
{3.5ex plus 1ex minus .2ex}{2.3ex plus .2ex}%
{\normalfont\Large\bfseries\raggedright}}%

%оформление подразделов
\renewcommand{\subsection}{\@startsection{subsection}{2}{18pt}%
{3.25ex plus 1ex minus .2ex}{1.5ex plus .2ex}%
{\normalfont\large\bfseries\raggedright}}%

%оформление подподразделов
\renewcommand{\subsubsection}{\@startsection{subsubsection}{3}{18pt}%
{3.25ex plus 1ex minus .2ex}{1.5ex plus .2ex}%
{\normalfont\large\bfseries\raggedright}}%

%оформление библиографии
\bibliographystyle{unsrt} 
%\renewcommand{\@biblabel}[1]{#1.}

\addto\captionsrussian{\renewcommand\figurename{Рисунок}}
\addto\captionsrussian{\renewcommand\figurename{Рисунок}}

%Оформление подписи рисунка
\renewcommand \thefigure{\@arabic\c@figure}
%чтобы номер рисунка содержал номер главы (например, рисунок 2.1) надо закомментировать предыдущую строку
\renewenvironment{figure}{%
\let\@makecaption\@makefigurecaption
\@float{figure}}%
{%
\addtocontents{lof}{ {\vskip 0.4em} }%
\end@float%
}
%

%Оформление подписи таблицы
\newcommand{\@makefigurecaption}[2]{%
\vspace{\abovecaptionskip}%
\sbox{\@tempboxa}{\large #1 --- \large #2}%
\ifdim \wd\@tempboxa >\hsize {\center\hyphenpenalty=10000\large #1 --- \large #2 \par}%
\else \global\@minipagefalse \hbox to \hsize
{\hfil \hyphenpenalty=10000 \large #1 --- \large #2\hfil}%
\fi \vspace{\belowcaptionskip}}


\renewcommand{\thetable}{\@arabic\c@table}
%чтобы номер таблицы содержал номер главы (например, таблица 2.1) надо закомментировать предыдущую строку
\renewenvironment{table}{%
\let\@makecaption\@maketablecaption
\@float{table}}%
{%
\addtocontents{lot}{ {\vskip 0.4em} }%
\end@float%
}
%

\newlength\abovetablecaptionskip
\newlength\belowtablecaptionskip
\newlength\tableparindent
\setlength\abovetablecaptionskip{10\p@}
\setlength\belowtablecaptionskip{0\p@}
\setlength\tableparindent{18\p@}
\newcommand{\@maketablecaption}[2]{
  \vskip\abovetablecaptionskip
  \hskip\tableparindent \large #1~---\ \large #2\par
  \vskip\belowtablecaptionskip
}

\makeatother%

\sloppy