\documentclass[a4paper,12pt]{report}

\usepackage[russian]{babel}
\usepackage[utf8]{inputenc}
\usepackage{amssymb,amsthm,amsmath,amscd}
\usepackage{graphicx}
\usepackage{float}
\usepackage{cite}
%пакет cite служит для простой вставки ссылок на публикации в работе, если что,
%вызов этого пакета можно закомментировать с помощью символа %
\usepackage{longtable}

\usepackage{listings}
% Злостный мегакостыль для поддержки кириллицы в листингах
\lstset{
literate={а}{{\selectfont\char224}}1
{б}{{\selectfont\char225}}1
{в}{{\selectfont\char226}}1
{г}{{\selectfont\char227}}1
{д}{{\selectfont\char228}}1
{е}{{\selectfont\char229}}1
{ё}{{\"e}}1
{ж}{{\selectfont\char230}}1
{з}{{\selectfont\char231}}1
{и}{{\selectfont\char232}}1
{й}{{\selectfont\char233}}1
{к}{{\selectfont\char234}}1
{л}{{\selectfont\char235}}1
{м}{{\selectfont\char236}}1
{н}{{\selectfont\char237}}1
{о}{{\selectfont\char238}}1
{п}{{\selectfont\char239}}1
{р}{{\selectfont\char240}}1
{с}{{\selectfont\char241}}1
{т}{{\selectfont\char242}}1
{у}{{\selectfont\char243}}1
{ф}{{\selectfont\char244}}1
{х}{{\selectfont\char245}}1
{ц}{{\selectfont\char246}}1
{ч}{{\selectfont\char247}}1
{ш}{{\selectfont\char248}}1
{щ}{{\selectfont\char249}}1
{ъ}{{\selectfont\char250}}1
{ы}{{\selectfont\char251}}1
{ь}{{\selectfont\char252}}1
{э}{{\selectfont\char253}}1
{ю}{{\selectfont\char254}}1
{я}{{\selectfont\char255}}1
{А}{{\selectfont\char192}}1
{Б}{{\selectfont\char193}}1
{В}{{\selectfont\char194}}1
{Г}{{\selectfont\char195}}1
{Д}{{\selectfont\char196}}1
{Е}{{\selectfont\char197}}1
{Ё}{{\"E}}1
{Ж}{{\selectfont\char198}}1
{З}{{\selectfont\char199}}1
{И}{{\selectfont\char200}}1
{Й}{{\selectfont\char201}}1
{К}{{\selectfont\char202}}1
{Л}{{\selectfont\char203}}1
{М}{{\selectfont\char204}}1
{Н}{{\selectfont\char205}}1
{О}{{\selectfont\char206}}1
{П}{{\selectfont\char207}}1
{Р}{{\selectfont\char208}}1
{С}{{\selectfont\char209}}1
{Т}{{\selectfont\char210}}1
{У}{{\selectfont\char211}}1
{Ф}{{\selectfont\char212}}1
{Х}{{\selectfont\char213}}1
{Ц}{{\selectfont\char214}}1
{Ч}{{\selectfont\char215}}1
{Ш}{{\selectfont\char216}}1
{Щ}{{\selectfont\char217}}1
{Ъ}{{\selectfont\char218}}1
{Ы}{{\selectfont\char219}}1
{Ь}{{\selectfont\char220}}1
{Э}{{\selectfont\char221}}1
{Ю}{{\selectfont\char222}}1
{Я}{{\selectfont\char223}}1
}

%Оформление теорем, лемм и т.д.
\theoremstyle{plain} % default
\newtheorem{Theorem}{Теорема}[chapter]
\newtheorem{Lemma}{Лемма}[chapter]
\newtheorem{Proposition}{Предложение}[chapter]
\newtheorem{Corollary}{Следствие}[chapter]
\newtheorem{Statement}{Утверждение}[chapter]

\theoremstyle{definition}
\newtheorem{Definition}{Определение}[chapter]
\newtheorem{Conjecture}{Гипотеза}[chapter]
\newtheorem{Algorithm}{Алгоритм}[chapter]
\newtheorem{Property}{Свойство}[chapter]

\theoremstyle{remark}
\newtheorem{Remark}{Замечание}[chapter]
\newtheorem{Example}{Пример}[chapter]
\newtheorem{Note}{Примечание}[chapter]
\newtheorem{Case}{Случай}[chapter]

%Определение полей
\setlength{\oddsidemargin}{0cm}% 1in=2.54см
\setlength{\hoffset}{0.46cm}% 1in+\hoffset=3cm = левое поле;

\setlength{\textwidth}{17cm}% 21cm-3cm(левое поле)-1cm(правое поле)=17cm;

\setlength{\headheight}{0cm}%
\setlength{\topmargin}{0cm}%
\setlength{\headsep}{0cm}%
\setlength{\voffset}{-0.54cm}% 1in+\voffset=2cm = верхнее поле;

\setlength{\textheight}{25.7cm}% 29.7cm-2cm(верхнее поле)-2cm(нижнее поле)=25.7cm;

%Оформление глав, разделов и т.д.
\makeatletter%

%не подавлять абзацный отступ в главах
\renewcommand{\chapter}{\cleardoublepage\thispagestyle{plain}%
\global\@topnum=0 \@afterindenttrue \secdef\@chapter\@schapter}

%оформление нумерованных глав
\renewcommand{\@makechapterhead}[1]{%Начало макроопределения
\vspace*{50pt}%Пустое место вверху страницы
{\parindent=18pt \normalfont\Large\bfseries
\thechapter{} %номер главы
\normalfont\Large\bfseries #1 \par %заголовок от текста
\nopagebreak %чтоб не оторвать заголовок от текста
\vspace{40 pt} %между заголовком и текстом
}%конец группы
}%коней макроопределения

%оформление ненумерованных глав
\renewcommand{\@makeschapterhead}[1]{%Начало макроопределения
\vspace*{50pt}%Пустое место вверху страницы
{\parindent=18pt \normalfont\Large\bfseries #1 \par %заголовок от текста
\nopagebreak %чтоб не оторвать заголовок от текста
\vspace{40pt} %между заголовком и текстом
}%конец группы
}%коней макроопределения

%оформление разделов
\renewcommand{\section}{\@startsection{section}{1}{18pt}%
{3.5ex plus 1ex minus .2ex}{2.3ex plus .2ex}%
{\normalfont\Large\bfseries\raggedright}}%

%оформление подразделов
\renewcommand{\subsection}{\@startsection{subsection}{2}{18pt}%
{3.25ex plus 1ex minus .2ex}{1.5ex plus .2ex}%
{\normalfont\large\bfseries\raggedright}}%

%оформление подподразделов
\renewcommand{\subsubsection}{\@startsection{subsubsection}{3}{18pt}%
{3.25ex plus 1ex minus .2ex}{1.5ex plus .2ex}%
{\normalfont\large\bfseries\raggedright}}%

%оформление библиографии
\bibliographystyle{unsrt} 
%\renewcommand{\@biblabel}[1]{#1.}

\addto\captionsrussian{\renewcommand\figurename{Рисунок}}
\addto\captionsrussian{\renewcommand\figurename{Рисунок}}

%Оформление подписи рисунка
%\renewcommand \thefigure{\@arabic\c@figure}
%чтобы номер рисунка содержал номер главы (например, рисунок 2.1) надо закомментировать предыдущую строку
\renewenvironment{figure}{%
\let\@makecaption\@makefigurecaption
\@float{figure}}%
{%
\addtocontents{lof}{ {\vskip 0.4em} }%
\end@float%
}
%

%Оформление подписи таблицы
\newcommand{\@makefigurecaption}[2]{%
\vspace{\abovecaptionskip}%
\sbox{\@tempboxa}{\large #1 --- \large #2}%
\ifdim \wd\@tempboxa >\hsize {\center\hyphenpenalty=10000\large #1 --- \large #2 \par}%
\else \global\@minipagefalse \hbox to \hsize
{\hfil \hyphenpenalty=10000 \large #1 --- \large #2\hfil}%
\fi \vspace{\belowcaptionskip}}


%\renewcommand{\thetable}{\@arabic\c@table}
%чтобы номер таблицы содержал номер главы (например, таблица 2.1) надо закомментировать предыдущую строку
\renewenvironment{table}{%
\let\@makecaption\@maketablecaption
\@float{table}}%
{%
\addtocontents{lot}{ {\vskip 0.4em} }%
\end@float%
}
%

\newlength\abovetablecaptionskip
\newlength\belowtablecaptionskip
\newlength\tableparindent
\setlength\abovetablecaptionskip{10\p@}
\setlength\belowtablecaptionskip{0\p@}
\setlength\tableparindent{18\p@}
\newcommand{\@maketablecaption}[2]{
  \vskip\abovetablecaptionskip
  \hskip\tableparindent \large #1~---\ \large #2\par
  \vskip\belowtablecaptionskip
}

\makeatother%

\sloppy