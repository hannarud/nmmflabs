\documentclass[a4paper,12pt]{report}

\usepackage[russian]{babel}
\usepackage[utf8]{inputenc}
\usepackage{amssymb,amsthm,amsmath,amscd}
\usepackage{graphicx}
\usepackage{float}
\usepackage{cite}
%пакет cite служит для простой вставки ссылок на публикации в работе, если что,
%вызов этого пакета можно закомментировать с помощью символа %

%Оформление теорем, лемм и т.д.
\theoremstyle{plain} % default
\newtheorem{Theorem}{Теорема}[chapter]
\newtheorem{Lemma}{Лемма}[chapter]
\newtheorem{Proposition}{Предложение}[chapter]
\newtheorem{Corollary}{Следствие}[chapter]
\newtheorem{Statement}{Утверждение}[chapter]

\theoremstyle{definition}
\newtheorem{Definition}{Определение}[chapter]
\newtheorem{Conjecture}{Гипотеза}[chapter]
\newtheorem{Algorithm}{Алгоритм}[chapter]
\newtheorem{Property}{Свойство}[chapter]

\theoremstyle{remark}
\newtheorem{Remark}{Замечание}[chapter]
\newtheorem{Example}{Пример}[chapter]
\newtheorem{Note}{Примечание}[chapter]
\newtheorem{Case}{Случай}[chapter]

%Определение полей
\setlength{\oddsidemargin}{0cm}% 1in=2.54см
\setlength{\hoffset}{0.46cm}% 1in+\hoffset=3cm = левое поле;

\setlength{\textwidth}{17cm}% 21cm-3cm(левое поле)-1cm(правое поле)=17cm;

\setlength{\headheight}{0cm}%
\setlength{\topmargin}{0cm}%
\setlength{\headsep}{0cm}%
\setlength{\voffset}{-0.54cm}% 1in+\voffset=2cm = верхнее поле;

\setlength{\textheight}{25.7cm}% 29.7cm-2cm(верхнее поле)-2cm(нижнее поле)=25.7cm;

%Оформление глав, разделов и т.д.
\makeatletter%

%не подавлять абзацный отступ в главах
\renewcommand{\chapter}{\cleardoublepage\thispagestyle{plain}%
\global\@topnum=0 \@afterindenttrue \secdef\@chapter\@schapter}

%оформление нумерованных глав
\renewcommand{\@makechapterhead}[1]{%Начало макроопределения
\vspace*{50pt}%Пустое место вверху страницы
{\parindent=18pt \normalfont\Large\bfseries
\thechapter{} %номер главы
\normalfont\Large\bfseries #1 \par %заголовок от текста
\nopagebreak %чтоб не оторвать заголовок от текста
\vspace{40 pt} %между заголовком и текстом
}%конец группы
}%коней макроопределения

%оформление ненумерованных глав
\renewcommand{\@makeschapterhead}[1]{%Начало макроопределения
\vspace*{50pt}%Пустое место вверху страницы
{\parindent=18pt \normalfont\Large\bfseries #1 \par %заголовок от текста
\nopagebreak %чтоб не оторвать заголовок от текста
\vspace{40pt} %между заголовком и текстом
}%конец группы
}%коней макроопределения

%оформление разделов
\renewcommand{\section}{\@startsection{section}{1}{18pt}%
{3.5ex plus 1ex minus .2ex}{2.3ex plus .2ex}%
{\normalfont\Large\bfseries\raggedright}}%

%оформление подразделов
\renewcommand{\subsection}{\@startsection{subsection}{2}{18pt}%
{3.25ex plus 1ex minus .2ex}{1.5ex plus .2ex}%
{\normalfont\large\bfseries\raggedright}}%

%оформление подподразделов
\renewcommand{\subsubsection}{\@startsection{subsubsection}{3}{18pt}%
{3.25ex plus 1ex minus .2ex}{1.5ex plus .2ex}%
{\normalfont\large\bfseries\raggedright}}%

%оформление библиографии
\bibliographystyle{unsrt} 
%\renewcommand{\@biblabel}[1]{#1.}

\addto\captionsrussian{\renewcommand\figurename{Рисунок}}
\addto\captionsrussian{\renewcommand\figurename{Рисунок}}

%Оформление подписи рисунка
\renewcommand \thefigure{\@arabic\c@figure}
%чтобы номер рисунка содержал номер главы (например, рисунок 2.1) надо закомментировать предыдущую строку
\renewenvironment{figure}{%
\let\@makecaption\@makefigurecaption
\@float{figure}}%
{%
\addtocontents{lof}{ {\vskip 0.4em} }%
\end@float%
}
%

%Оформление подписи таблицы
\newcommand{\@makefigurecaption}[2]{%
\vspace{\abovecaptionskip}%
\sbox{\@tempboxa}{\large #1 --- \large #2}%
\ifdim \wd\@tempboxa >\hsize {\center\hyphenpenalty=10000\large #1 --- \large #2 \par}%
\else \global\@minipagefalse \hbox to \hsize
{\hfil \hyphenpenalty=10000 \large #1 --- \large #2\hfil}%
\fi \vspace{\belowcaptionskip}}


\renewcommand{\thetable}{\@arabic\c@table}
%чтобы номер таблицы содержал номер главы (например, таблица 2.1) надо закомментировать предыдущую строку
\renewenvironment{table}{%
\let\@makecaption\@maketablecaption
\@float{table}}%
{%
\addtocontents{lot}{ {\vskip 0.4em} }%
\end@float%
}
%

\newlength\abovetablecaptionskip
\newlength\belowtablecaptionskip
\newlength\tableparindent
\setlength\abovetablecaptionskip{10\p@}
\setlength\belowtablecaptionskip{0\p@}
\setlength\tableparindent{18\p@}
\newcommand{\@maketablecaption}[2]{
  \vskip\abovetablecaptionskip
  \hskip\tableparindent \large #1~---\ \large #2\par
  \vskip\belowtablecaptionskip
}

\makeatother%

\sloppy

\begin{document}

\thispagestyle{empty}
\begin{normalsize}
\begin{center}
{\bf МИНИСТЕРСТВО ОБРАЗОВАНИЯ РЕСПУБЛИКИ БЕЛАРУСЬ}
\end{center}

\begin{center}
{\bf БЕЛОРУССКИЙ ГОСУДАРСТВЕННЫЙ УНИВЕРСИТЕТ}
\end{center}

\begin{center}
{\bf Факультет прикладной математики и информатики}
\end{center}

\begin{center}
Кафедра математического моделирования и анализа данных
\end{center}
\end{normalsize}
\bigskip
\bigskip
\bigskip
\bigskip
\bigskip
\bigskip

\begin{center}
{\bf РУДАКОВСКАЯ АННА ВЯЧЕСЛАВОВНА}
\end{center}
\bigskip

\begin{center}
{\bf ЛАБОРАТОРНАЯ РАБОТА №1. ПОСТРОЕНИЕ И РЕАЛИЗАЦИЯ РАЗНОСТНЫХ СХЕМ ДЛЯ ТРЕТЬЕЙ КРАЕВОЙ ЗАДАЧИ УРАВНЕНИЯ ТЕПЛОПРОВОДНОСТИ}
\end{center}
\bigskip
\bigskip
\bigskip
\bigskip

\begin{center}
Отчет о лабораторной работе\linebreak
студентки 4 курса 7 группы
\end{center}
\bigskip
\bigskip
\bigskip
\bigskip
%\begin{flushleft}
\linespread{1.0}
\begin{tabular}{@{}p{9cm}@{}p{7cm}}
{\small Работа зачтена:} & {\bf\small Руководитель}\\
{\small{\bf }} & {\small Радкевич Елена Владимировна }\\
\end{tabular}
%\end{flushleft}

\vspace{\stretch{1}}

\begin{center}
\bf{МИНСК 2013}
\end{center}

\chapter{Постановка задачи}
Построить и реализовать разностную схему для уравнения теплопроводности (третья краевая задача) с весами $\sigma=0$, $\sigma=\frac{1}{2}$, $\sigma=1$ с порядком аппроксимации $O(\tau+h^2)$ (повысить порядок аппроксимации краевых условий).\par
Шаги сетки выбрать из условий устойчивости разностной схемы:
\begin{tabular}{cc}
$\sigma=0$ & $\tau \le \frac{h^{2}}{2}$ \\
$\sigma=\frac{1}{2}$ & $\tau \le h^{2}$ \\
$\sigma=\frac{1}{2}$ & $\tau \le h^{2}$
\end{tabular} \par
Отрезок интегрирования $0 \le \tau \le 1$ (для $\sigma=0$, $\tau = 0.1$). \par
Вывод результатов в одной таблице (каждый 20-й слой).

\chapter{Описание алгоритмов}
\section{Общая постановка задачи и построение разностной схемы}
Ищем приближенное решение для задачи вида:
\begin{equation}
\left\{
\begin{array}{ll}
\frac{\partial u}{\partial t}=\frac{\partial^{2} u}{\partial x^{2}}+f(x,t), & 0<x<1, 0<t\le T, \\
u(x,0)=u_{0}(x), & 0\le x \le 1, \\
\frac{\partial u}{\partial x}(0,t)=\alpha_{1}u(0,t)+\beta_{1}(t), & 0 \le t \le T, \\
\frac{\partial u}{\partial x}(1,t)=\alpha_{2}u(1,t)+\beta_{2}(t), & 0 \le t \le T.
\end{array}
\right.
\end{equation}

Построим эту задачу при $u(x,t)=e^{-t}(\sin x + \cos x)$.
\begin{eqnarray*}
& \frac{\partial u}{\partial t}=-e^{-t}(\sin x + \cos x) \\
& \frac{\partial u}{\partial x}=e^{-t}(\cos x - \sin x) \\
& \frac{\partial^{2} u}{\partial x^{2}}=e^{-t}(-\sin x - \cos x) \\
& f(x,t)=\frac{\partial u}{\partial t}-\frac{\partial^{2} u}{\partial x^{2}}=e^{-t}(-\sin x - \cos x + \sin x + \cos x)=0 \\
& u_{0}(x)=\sin x + \cos x
\end{eqnarray*}


%
%
%Общий вид уравнения:
%\begin{equation}\label{eq:eq1}
%\ u(x)-\lambda\int_a^bK(x,s)u(s)ds = f(x)
%\end{equation}
%На отрезке $[a;b]$ выбираем $n+1$ узел по правилу
%\begin{equation*}
%\ h=\frac{b-a}{n}, x_k=a+hk, k=\overline{0,n}
%\end{equation*}
%По этим узлам строим КФ, которой заменяем интеграл и получаем:
%\begin{equation}\label{eq:eq2}
%\ u(x)-\lambda\sum_{k=0}^{n}A_kK(x,x_k)u(x_k) = f(x)
%\end{equation}
%Обозначив $y_k=u(x_k)$, приходим у СЛАУ относительно $y_i$:
%\begin{equation}\label{eq:eq3}
%\ y_i-\lambda\sum_{k=0}^{n}A_kK(x_i,x_k)y_k = f(x_i), i=\overline{0,n}
%\end{equation}
%Решив эту СЛАУ, восстанавливаем решение исходной задачи в виде
%\begin{equation}\label{eq:eq4}
%\ u(x)= f(x)+\lambda\sum_{k=0}^{n}A_kK(x,x_k)y_k
%\end{equation}
%\par
%Если в качестве КФ используем формулу трапеций, то $A_0=A_n=\frac{h}{2}, A_k=h, k=\overline{1,n-1}$ \par
%Если в качестве КФ используем формулу Симпсона, то $A_0=A_n=\frac{h}{3}, A_{2k}=\frac{2h}{3}, A_{2k-1}=\frac{4h}{3}, k=\overline{1,[\frac{n}{2}]}$
%
%\section{Метод механических квадратур для решения ИУ Вольтерра 2-го рода}
%Общий вид уравнения:
%\begin{equation}\label{eq:eq5}
%\ u(x)-\lambda\int_a^xK(x,s)u(s)ds = f(x), x\in[a;b]
%\end{equation}
%При помощи аналогичных рассуждений получаем СЛАУ относительно $y_i$:
%\begin{equation}\label{eq:eq6}
%\ y_i-\lambda\sum_{k=0}^{i}A_kK(x_i,x_k)y_k = f(x_i), i=\overline{0,n}
%\end{equation}
%(заметим, что для решения этой СЛАУ остаточно выполнить только обратный ход метода Гаусса, так как она уже имеет нижнетреугольный вид). В остальном алгортмы совпадают.
%
%\section{Метод замены ядра на вырожденное для решения ИУ Фредгольма 2-го рода}
%Метод применяется только к ИУ Фредгольма 2-го рода. \par
%Пусть имеется уравнение вида ~\eqref{eq:eq1}. Представляем ядро оператора Фредгольма $K(x,y)$ в виде
%\begin{equation}\label{eq:eq7}
%\ K(x,y)=\sum_{i=0}^n\alpha_i(x)\beta_i(y)
%\end{equation}
%где $\alpha_i(x), \beta_i(y), i=\overline{0,n}$ - коэффициенты в разложении функции $K(x,y)$, полученные каким-либо способом (разожение в ряд Тейлора по одной из переменных, по обеим, разложение в ряд Фурье, интерполирование, метод Бэтмена).
%При помощи такой замены сразу же приходим к алгоритму:
%\begin{equation}\label{eq:eq8}
%\ u(x)= f(x)+\lambda\sum_{i=0}^{n}c_i\alpha_i(x)
%\end{equation}
%где $c_i$ - это решение СЛАУ
%\begin{equation}\label{eq:eq9}
%\ c_i=\int_a^b\beta_i(s)f(s)ds+\lambda\sum_{j=0}^nc_j\int_a^b\beta_i(s)\alpha_j(s)ds, i=\overline{0,n}
%\end{equation}
%
%\chapter{Полученные результаты}
%\section{ММК для ИУ Фредгольма 2-го рода, используя формулу трапеций}
%Приближенное решение:
%\begin{eqnarray*}
%\ u(x)=f(x)+\lambda*0.15708*K(x,0)*2+ 0.314159*K(x,0.314159)*3.15999+ \\ +0.314159*K(x,0.628319)*5.38522+ 0.314159*K(x,0.942478)*8.5126+ \\ +0.314159*K(x,1.25664)*11.8694+ 0.314159*K(x,1.5708)*14.1561+ \\ +0.314159*K(x,1.88496)*14.3886+ 0.314159*K(x,2.19911)*13.9291+ \\ +0.314159*K(x,2.51327)*16.4773+ 0.314159*K(x,2.82743)*22.9691
%\end{eqnarray*}
%График полученного приближенного решения: \par
%%\includegraphics[width=10cm, keepaspectratio]{Lab5_MMK_Fredholm_Trapeze.png}
%
%\section{ММК для ИУ Фредгольма 2-го рода, используя формулу Симпсона}
%Приближенное решение:
%\begin{eqnarray*}
%\ u(x)=f(x)+\lambda*0.10472*K(x,0)*2+ 0.418879*K(x,0.314159)*3.1531+ \\ +0.20944*K(x,0.628319)*5.35899+ 0.418879*K(x,0.942478)*8.46566+ \\ +0.20944*K(x,1.25664)*11.8347+ 0.418879*K(x,1.5708)*14.2053+ \\ +0.20944*K(x,1.88496)*14.5356+ 0.418879*K(x,2.19911)*13.9709+ \\ +0.20944*K(x,2.51327)*16.2136+ 0.418879*K(x,2.82743)*22.8794
%\end{eqnarray*}
%График полученного приближенного решения: \par
%%\includegraphics[width=10cm, keepaspectratio]{Lab5_MMK_Fredholm_Simpson.png}
%
%\section{Метод замены ядра на вырожденное для ИУ Фредгольма 2-го рода}
%Для замены на вырожденное используем интерполирование ядра $K(x,y)=sin\frac{x^2y^2}{10}$ по переменной $x$.
%Подставляем функцию из условия, находим коэффициенты в разложении:
%\begin{equation*}
%\begin{aligned}
%\alpha_0(x)=\frac{(x-\frac{\pi}{3})(x-\frac{2\pi}{3})(x-\pi)}{(0-\frac{\pi}{3})(0-\frac{2\pi}{3})(0-\pi)} \\
%\alpha_1(x)=\frac{x(x-\frac{2\pi}{3})(x-\pi)}{\frac{\pi}{3}(\frac{\pi}{3}-\frac{2\pi}{3})(\frac{\pi}{3}-\pi)} \\
%\alpha_2(x)=\frac{x(x-\frac{\pi}{3})(x-\pi)}{\frac{2\pi}{3}(\frac{2\pi}{3}-\frac{\pi}{3})(\frac{2\pi}{3}-\pi)} \\
%\alpha_3(x)=\frac{x(x-\frac{\pi}{3})(x-\frac{2\pi}{3})}{\pi(\pi-\frac{\pi}{3})(\pi-\frac{2\pi}{3})} \\
%\beta_0(y)=K(0,y) \\
%\beta_1(y)=K(\frac{\pi}{3},y) \\
%\beta_2(y)=K(\frac{2\pi}{3},y) \\
%\beta_3(y)=K(\pi,y) \\
%\end{aligned}
%\end{equation*}
%Считаем интегралы и получаем СЛАУ относительно коэффициентов $c_i$, решаем ее. Приближенное решение будет иметь вид:
%\begin{eqnarray*}
%\ u(x)=f(x)+\lambda(0\alpha_0(x)+55.8958\alpha_1(x)- \\
%-40.0692\alpha_2(x)-38.8226\alpha_3(x)
%\end{eqnarray*}
%График полученного приближенного решения: \par
%%\includegraphics[width=10cm, keepaspectratio]{Lab5_MMK_Fredholm_Degenerate.png}
%
%\section{Сравнение результатов применения различных методов к ИУ Фредгольма 2-го рода}
%В таблице содержатся значения в узлах интерполирования, полученные при помощи различных методов (ММК с формулой трапеций(ММКТ), ММК с формулой Симпсона(ММКС), метод замены ядра на вырожденное(ЗЯВ)): \par
%\begin{tabular}{lccr}
%Узел &ММКТ &ММКС &ЗЯВ \\
%0 &2 &2 &2 \\
%0.314159 &3.15999 &3.1531 &14.5668 \\
%0.628319 &5.38522 &5.35899 &19.93 \\
%0.942478 &8.5126 &8.46566 &19.8212 \\
%1.25664 &11.8694 &11.8347 &15.9906 \\
%1.5708 &14.1561 &14.2053 &10.2135 \\
%1.88496 &14.3886 &14.5356 &4.30033 \\
%2.19911 &13.9291 &13.9709 &0.108966 \\
%2.51327 &16.4773 &16.2136 &-0.437435 \\
%2.82743 &22.9691 &22.8794 &4.67356
%\end{tabular} \par
%Таким образом, погрешность метода замены ядра на вырожденное получается очень большой (из-за вида ядра, которое, по сути, представляет собой неберущийся интеграл). В данном случае использовать этот метод крайне неэффективно
%
%\section{ММК для ИУ Вольтерра 2-го рода, используя формулу трапеций}
%Приближенное решение:
%\begin{eqnarray*}
%\ u(x)=f(x)+\lambda*0.05*K(x,0)*2+ 0.1*K(x,0.1)*2.65637+ 0.1*K(x,0.2)*3.49374+ \\ +0.1*K(x,0.3)*4.80492+ 0.1*K(x,0.4)*6.96753+ 0.1*K(x,0.5)*10.7857+ \\ +0.1*K(x,0.6)*18.1602+ 0.1*K(x,0.7)*34.233+ 0.1*K(x,0.8)*75.688+ \\ +0.1*K(x,0.9)*212.94
%\end{eqnarray*}
%Значения в узлах интерполирования: \par
%\begin{tabular}{lcr}
%Узел &Значение \\
%0 &136.564 \\
%0.1 &150.937 \\
%0.2 &166.849 \\
%0.3 &184.464 \\
%0.4 &203.971 \\
%0.5 &225.577 \\
%0.6 &249.511 \\
%0.7 &276.034 \\
%0.8 &305.43 \\
%0.9 &338.023
%\end{tabular} \par
%График полученного приближенного решения: \par
%%\includegraphics[width=10cm, keepaspectratio]{Lab5_MMK_Volterr_Trapeze.png}
%
%\chapter{Листинг программы}
%\begin{verbatim}
%/*
% * main.cpp
% *
% *  Created on: 10.04.2013
% *      Author: HR
% */
%
%#include <iostream>
%#include <math.h>
%
%using namespace std;
%
%void straightGaussianElimination (double ** A, double * b, int size)
%{
%	double c;
%	double d;
%
%	// partial pivoting
%	for (int i = size-1; i > 0; i--)
%    {
%        if ( A[i-1][0] < A[i][0] )
%		{
%            for( int j = 0; j < size; j++ )
%            {
%                c = A[i][j];
%                A[i][j] = A[i-1][j];
%                A[i-1][j] = c;
%            }
%			d = b[i];
%			b[i] = b[i-1];
%			b[i-1] = d;
%		}
%    }
%
%	for ( int k = 0; k < size; k++ )
%	{
%		c = A[k][k];
%		A[k][k] = 1;
%		b[k] = b[k]/c;
%		for ( int j = k+1; j < size; j++ )
%		{
%			A[k][j] = A[k][j] / c;
%		}
%		for (int i = k+1; i < size; i++)
%		{
%			for (int j = k+1; j < size; j++)
%			{
%				A[i][j] = A[i][j] - A[i][k]*A[k][j];
%			}
%			b[i] = b[i] - A[i][k]*b[k];
%		}
%	}
%	b[size-1] = b[size-1]/A[size-1][size-1];
%}
%
%void inverseGaussianElimination (double * result, double ** A, double * b, int size)
%{
%	result[size-1] = b[size-1];
%	for (int i = size-2; i >= 0; i--)
%	{
%		double sum = 0;
%		for (int j = i+1; j < size; j++)
%		{
%			sum = sum + A[i][j]*result[j];
%		}
%		result[i] = b[i] - sum;
%	}
%}
%
%void GaussianElimination (double * result, double ** A, double * b, int size)
%{
%	straightGaussianElimination ( A, b, size );
%	inverseGaussianElimination ( result, A, b, size );
%}
%
%void lowerTriangularGaussianElimination (double * result, double ** A, double * b, int size)
%{
%	result[0] = b[0];
%	for (int i = 1; i < size; i++)
%	{
%		double sum = 0;
%		for (int j = 0; j < i; j++)
%		{
%			sum = sum + A[i][j]*result[j];
%		}
%		result[i] = (b[i] - sum)/A[i][i];
%	}
%}
%
%double K_Fredholm(double x, double y)
%{
%	return sin(pow(x*y, 2)/10);
%}
%
%double f_Fredholm(double x)
%{
%	return 1+x+exp(x);
%}
%
%double K_Volterr(double x, double y)
%{
%	return exp(x+y);
%}
%
%double f_Volterr(double x)
%{
%	return 1+exp(2*x);
%}
%
%double MMKFredholmTrapeze(double x)
%{
%	//number of nodes
%	int n = 10;
%	//equation input
%	double a = 0;
%	double b = M_PI;
%	double lambda = 0.25;
%	double (*f)(double) = f_Fredholm;
%	double (*K)(double, double) = K_Fredholm;
%
%	//coefficients for Trapeze SF
%	double h = (a+b)/n;
%	double *A = new double[n+1];
%	A[0] = A[n] = h/2;
%	for (int i = 1; i < n; i++) //from second to the pre-last
%	{
%		A[i] = h;
%	}
%
%	double **B = new double*[n+1];
%	for(int i = 0; i < n+1; i++)
%	{
%		B[i] = new double [n+1];
%	}
%	double *c = new double[n+1];
%	double *y = new double[n+1]; //for the answer
%
%	//composing Fredholm Linear System
%	for(int i = 0; i < n+1; i++)
%	{
%		for(int j = 0; j < n+1; j++)
%		{
%			if (i != j)
%				B[i][j] = -lambda*A[j]*K(a+h*i, a+h*j);
%			else
%				B[i][j] = 1 - lambda*A[j]*K(a+h*i, a+h*j);
%		}
%		c[i] = f(a+h*i);
%	}
%
%	//solving the system
%	GaussianElimination (y, B, c, n+1);
%
%	double functionMeaning;
%
%	cout<<"Approximate u(x) is:"<<endl;
%	cout<<"u(x)=f(x)+lambda*";
%	for(int i = 0; i < n; i++)
%	{
%		cout<<A[i]<<"*K(x,"<<a+h*i<<")*"<<y[i]<<"+ ";
%	}
%	cout<<endl;
%
%	cout<<"Function meanings on the nodes:"<<endl;
%	for(int i = 0; i < n; i++)
%	{
%		functionMeaning = f(a+h*i);
%		for(int j = 0; j < n+1; j++)
%		{
%			functionMeaning += lambda*A[j]*K(a+h*i, a+h*j)*y[j];
%		}
%		cout<<a+h*i<<" "<<functionMeaning<<endl<<endl;
%	}
%	cout<<endl;
%
%	//composing the function meaning
%	functionMeaning = f(x);
%	for(int i = 0; i < n+1; i++)
%	{
%		functionMeaning += lambda*A[i]*K(x, a+h*i)*y[i];
%	}
%
%	delete [] A;
%	for(int i = 0; i < n+1; i++)
%	{
%		delete [] B[i];
%	}
%	delete [] B;
%	delete [] c;
%	delete [] y;
%
%	return functionMeaning;
%}
%
%double MMKFredholmSimpson(double x)
%{
%	//number of nodes
%	int n = 10;
%	//equation input
%	double a = 0;
%	double b = M_PI;
%	double lambda = 0.25;
%	double (*f)(double) = f_Fredholm;
%	double (*K)(double, double) = K_Fredholm;
%
%	//coefficients for Simpson SF
%	double h = (a+b)/n;
%	double *A = new double[n+1];
%	A[0] = A[n] = h/3;
%	for (int i = 1; i < n; i++) //from second to the pre-last
%	{
%		if (i%2 == 0)
%			A[i] = 2*h/3;
%		else
%			A[i] = 4*h/3;
%	}
%
%	double **B = new double*[n+1];
%	for(int i = 0; i < n+1; i++)
%	{
%		B[i] = new double [n+1];
%	}
%	double *c = new double[n+1];
%	double *y = new double[n+1]; //for the answer
%
%	//composing Fredholm Linear System
%	for(int i = 0; i < n+1; i++)
%	{
%		for(int j = 0; j < n+1; j++)
%		{
%			if (i != j)
%				B[i][j] = -lambda*A[j]*K(a+h*i, a+h*j);
%			else
%				B[i][j] = 1 - lambda*A[j]*K(a+h*i, a+h*j);
%		}
%		c[i] = f(a+h*i);
%	}
%
%	//solving the system
%	GaussianElimination (y, B, c, n+1);
%
%	cout<<"Approximate u(x) is:"<<endl;
%	cout<<"u(x)=f(x)+lambda*";
%	for(int i = 0; i < n; i++)
%	{
%		cout<<A[i]<<"*K(x,"<<a+h*i<<")*"<<y[i]<<"+ ";
%	}
%	cout<<endl;
%
%	double functionMeaning;
%
%	cout<<"Function meanings on the nodes:"<<endl;
%	for(int i = 0; i < n; i++)
%	{
%		functionMeaning = f(a+h*i);
%		for(int j = 0; j < n+1; j++)
%		{
%			functionMeaning += lambda*A[j]*K(a+h*i, a+h*j)*y[j];
%		}
%		cout<<a+h*i<<" "<<functionMeaning<<endl<<endl;
%	}
%	cout<<endl;
%
%	//composing the function meaning
%	functionMeaning = f(x);
%	for(int i = 0; i < n+1; i++)
%	{
%		functionMeaning += lambda*A[i]*K(x, a+h*i)*y[i];
%	}
%
%	delete [] A;
%	for(int i = 0; i < n+1; i++)
%	{
%		delete [] B[i];
%	}
%	delete [] B;
%	delete [] c;
%	delete [] y;
%
%	return functionMeaning;
%}
%
%double alpha0 (double x)
%{
%	return ((x - M_PI/3)*(x - 2*M_PI/3)*(x - M_PI))/((0 - M_PI/3)*(0 - 2*M_PI/3)*(0 - M_PI));
%}
%
%double alpha1 (double x)
%{
%	return ((x)*(x - 2*M_PI/3)*(x - M_PI))/((M_PI/3)*(M_PI/3 - 2*M_PI/3)*(M_PI/3 - M_PI));
%}
%
%double alpha2 (double x)
%{
%	return ((x)*(x - M_PI/3)*(x - M_PI))/((2*M_PI/3)*(2*M_PI/3 - M_PI/3)*(2*M_PI/3 - M_PI));
%}
%
%double alpha3 (double x)
%{
%	return ((x)*(x - M_PI/3)*(x - 2*M_PI/3))/((M_PI)*(M_PI - M_PI/3)*(M_PI - 2*M_PI/3));
%}
%
%double DegenerateKernelFredholm(double x)
%{
%	//number of nodes
%	int n = 10;
%	//equation input
%	double a = 0;
%	double end = M_PI;
%	double lambda = 0.25;
%	double (*f)(double) = f_Fredholm;
%
%	double (*a0)(double) = alpha0;
%	double (*a1)(double) = alpha1;
%	double (*a2)(double) = alpha2;
%	double (*a3)(double) = alpha3;
%
%	double h = (a+end)/n;
%
%	double **A = new double*[4];
%	for(int i = 0; i < 4; i++)
%	{
%		A[i] = new double[4];
%	}
%
%	A[0][0]=0.; A[0][1]=0.; A[0][2]=0.; A[0][3]=0.;
%	A[1][0]=0.0249107; A[1][1]=0.0159671; A[1][2]=0.708695; A[1][3]=0.292461;
%	A[2][0]=-0.0113004; A[2][1]=0.569942; A[2][2]=0.733566; A[2][3]=-0.180571;
%	A[3][0]=0.0403108; A[3][1]=0.702038; A[3][2]=-0.112338; A[3][3]=0.155404;
%	double *b = new double[4];
%	b[0]=0.;
%	b[1]=15.9218;
%	b[2]=6.14681;
%	b[3]=6.42797;
%	double *y = new double[4];
%
%	for(int i = 0; i < 4; i++)
%	{
%		A[i][i]=1-lambda*A[i][i];
%	}
%
%	//solving the system
%	GaussianElimination (y, A, b, 4);
%
%	double functionMeaning;
%
%	cout<<"Function meanings on the nodes:"<<endl;
%	for(int i = 0; i < n; i++)
%	{
%		functionMeaning = f(a+h*i)+lambda*(y[0]*a0(a+h*i)+y[1]*a1(a+h*i)+y[2]*a2(a+h*i)+y[3]*a3(a+h*i));
%		cout<<a+h*i<<" "<<functionMeaning<<endl<<endl;
%	}
%	cout<<endl;
%
%	//composing the function meaning
%	functionMeaning = f(x)+lambda*(y[0]*a0(x)+y[1]*a1(x)+y[2]*a2(x)+y[3]*a3(x));
%
%	for(int i = 0; i < n+1; i++)
%	{
%		delete [] A[i];
%	}
%	delete [] A;
%	delete [] b;
%	delete [] y;
%
%	return functionMeaning;
%}
%
%double MMKVolterrTrapeze(double x)
%{
%	//number of nodes
%	int n = 10;
%	//equation input
%	double a = 0;
%	double b = 1;
%	double lambda = 1;
%	double (*f)(double) = f_Volterr;
%	double (*K)(double, double) = K_Volterr;
%
%	//coefficients for Trapeze SF
%	double h = (a+b)/n;
%	double *A = new double[n+1];
%	A[0] = A[n] = h/2;
%	for (int i = 1; i < n; i++) //from second to the pre-last
%	{
%		A[i] = h;
%	}
%
%	double **B = new double*[n+1];
%	for(int i = 0; i < n+1; i++)
%	{
%		B[i] = new double [n+1];
%	}
%	double *c = new double[n+1];
%	double *y = new double[n+1]; //for the answer
%
%	//composing Volterr Linear System
%	for(int i = 0; i < n+1; i++)
%	{
%		for(int j = 0; j <= i; j++)
%		{
%			if (i != j)
%				B[i][j] = -lambda*A[j]*K(a+h*i, a+h*j);
%			else
%				B[i][j] = 1 - lambda*A[j]*K(a+h*i, a+h*j);
%		}
%		c[i] = f(a+h*i);
%	}
%
%	//solving the system
%	lowerTriangularGaussianElimination (y, B, c, n+1);
%
%	//composing the function meaning
%	double functionMeaning = f(x);
%	for(int i = 0; i < n+1; i++)
%	{
%		functionMeaning += lambda*A[i]*K(x, a+h*i)*y[i];
%	}
%
%	cout<<"Approximate u(x) is:"<<endl;
%	cout<<"u(x)=f(x)+lambda*";
%	for(int i = 0; i < n; i++)
%	{
%		cout<<A[i]<<"*K(x,"<<a+h*i<<")*"<<y[i]<<"+ ";
%	}
%	cout<<endl;
%
%	cout<<"Function meanings on the nodes:"<<endl;
%	for(int i = 0; i < n; i++)
%	{
%		functionMeaning = f(a+h*i);
%		for(int j = 0; j < n+1; j++)
%		{
%			functionMeaning += lambda*A[j]*K(a+h*i, a+h*j)*y[j];
%		}
%		cout<<a+h*i<<" "<<functionMeaning<<endl<<endl;
%	}
%	cout<<endl;
%
%	delete [] A;
%	for(int i = 0; i < n+1; i++)
%	{
%		delete [] B[i];
%	}
%	delete [] B;
%	delete [] c;
%	delete [] y;
%
%	return functionMeaning;
%}
%
%int main(void)
%{
%	double x;
%
%	for (x = 0; x <= M_PI; x += M_PI/100)
%		cout<<MMKFredholmTrapeze(x)<<endl;
%
%	for (x = 0; x <= M_PI; x += M_PI/100)
%		cout<<MMKFredholmSimpson(x)<<endl;
%
%	for (x = 0; x <= M_PI; x += M_PI/100)
%		cout<<DegenerateKernelFredholm(x)<<endl;
%
%	for (x = 0; x <= 1; x += 0.01)
%		cout<<MMKVolterrTrapeze(0)<<endl;
%
%	return 0;
%}
%\end{verbatim}

\end{document}